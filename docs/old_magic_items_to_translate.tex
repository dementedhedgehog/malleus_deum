
\section{Magic Items}

\begin{description}
\item[Life-Eater Dagger] two pronged dagger of Dark-elf make +3/+3. SHIT

\end{description}




Silver, Alchemical .. lycanthropes

Cold Iron +1 vs elves

Potion of gaseous form

Potion of polymorphism

Grobnung's Dagger of Vengeance
When you acquire Grobnung's Dagger you may name, or describe, an individual who has wronged you in some way. The dagger then becomes attuned to that individual.  You must believe that they deserve to be punished for harm that they have caused to you.  When weilded against your target, Grobnung's Dagger bursts into flame, it grants +2 to attack and damage rolls, and deals an additional 1d4 fire damage.  The dagger stays attuned to that individual until their death.


Drow Music Box
This item is a little box, about 3 inches a side.  It is decorated with spider web filligree and has a little handle that can be extended from the side.  When turned the box plays twinkle twinkle little star.  When the song is over there's a 1 in 6 channce that poisoned needles will spear rapidly out of the box and then as rapidly retract.


Bless'ed Weapons
+d6 on a crit vs undead and demons 



Deck of Illusions: This set of parchment cards is usually found in an ivory, leather, or wooden box. A full deck consists of thirty-four cards. When a card is drawn at random and thrown to the ground, a major image of a creature is formed. The figment lasts until dispelled. The illusory creature cannot move more than 30 feet away from where the card landed, but otherwise moves and acts as if it were real. At all times it obeys the desires of the character who drew the card. When the illusion is dispelled, the card becomes blank and cannot be used again. If the card is picked up, the illusion is automatically and instantly dispelled. The cards in a deck and the illusions they bring forth are summarized on the following table. (Use one of the first two columns to simulate the contents of a full deck using either ordinary playing cards or tarot cards.)

Playing Card 	Tarot Card 	Creature
Ace of hearts 	IV. The Emperor 	Red dragon
King of hearts 	Knight of swords 	Male human fighter and four guards
Queen of hearts 	Queen of staves 	Female human wizard
Jack of hearts 	King of staves 	Male human druid
Ten of hearts 	VII. The Chariot 	Cloud giant
Nine of hearts 	Page of staves 	Ettin
Eight of hearts 	Ace of cups 	Bugbear
Two of hearts 	Five of staves 	Goblin
Playing Card 	Tarot Card 	Creature
Ace of diamonds 	III. The Empress 	Glabrezu (demon)
King of diamonds 	Two of cups 	Male elf wizard and female apprentice
Queen of diamonds 	Queen of swords 	Half-elf ranger (female)
Jack of diamonds 	XIV. Temperance 	Harpy
Ten of diamonds 	Seven of staves 	Male half-orc barbarian
Nine of diamonds 	Four of pentacles 	Ogre mage
Eight of diamonds 	Ace of pentacles 	Gnoll
Two of diamonds 	Six of pentacles 	Kobold
Playing Card 	Tarot Card 	Creature
Ace of spades 	II. The High Priestess 	Lich
King of spades 	Three of staves 	Three male human clerics
Queen of spades 	Four of cups 	Medusa
Jack of spades 	Knight of pentacles 	Male dwarf paladin
Ten of spades 	Seven of swords 	Frost giant
Nine of spades 	Three of swords 	Troll
Eight of spades 	Ace of swords 	Hobgoblin
Two of spades 	Five of cups 	Goblin
Playing Card 	Tarot Card 	Creature
Ace of clubs 	VIII. Strength 	Iron golem
King of clubs 	Page of pentacles 	Three male halfling rogues
Queen of clubs 	Ten of cups 	Pixies
Jack of clubs 	Nine of pentacles 	Female half-elf bard
Ten of clubs 	Nine of staves 	Hill giant
Nine of clubs 	King of swords 	Ogre
Eight of clubs 	Ace of staves 	Orc
Two of clubs 	Five of cups 	Kobold
Playing Card 	Tarot Card 	Creature
Joker 	Two of pentacles 	Illusion of deck’s owner
Joker 	Two of staves 	Illusion of deck’s owner (sex reversed)

A randomly generated deck is usually complete (11–100 on d%), but may be discovered (01–10) with 1d20 of its cards missing. If cards are missing, reduce the price by a corresponding amount. 


Portable Hole: A portable hole is a circle of cloth spun from the webs of a phase spider interwoven with strands of ether. When opened fully, a portable hole is 3 feet in diameter, but it can be folded up to be as small as a pocket handkerchief. When spread upon any surface, it briefly sends matter under it to a pocket dimension.  After the material is removed or a few minutes have passed the hole disapears.  

Robe of the Beholder
Its wearer is able to see in all directions at the same moment due to scores of visible, magical eyelike patterns that adorn the robe. She also gains 20-foot darkvision. Bright lights may blind the robe for a short period of time.

Drow Bloodletter
+2 to damage (+0 to hit)
on a roll of 1 the bloodletter hits the wielder of the sword.


Cloak of Grounding..
Cloak with a houndstooth pattern on it.  Reduces lightning damage by 2 hit points per dice.  

Alarm Wire
A spool of fine copper wire on a wooden plug. When a piece of wire is snipped off and straightened, it may be placed across a hallway, windowsill, etc. and has the same effect when crossed over as an alarm spell. Each piece will function only once. An unused spool has 50 feet of wire.

Amber Spider
A small polished piece of amber with a spider trapped inside, held in a silver web pendant. The wearer of this pendant cannot be trapped in spider webs or web spells, and can climb across them at ½ movement rate. 10\% of these also allow the casting of a spider climb spell 1/day.

Amulet of Proof Against Turning
A bronze pendant of the sun being eclipsed by the moon. The pendant confers immunity to being turned or controlled. May be used by any person/creature subject to turning (i.e. undead, paladins, etc.) Useful also for a mage that has created undead and wishes to keep a priest from wresting away control. It must be worn by the creature/person that it will affect. The evil mage Ming is suspected of working on a version to be worn by a spellcaster to simultaneously protect any undead he has raised.

 

Bag of Briers
Inside this small, roughly woven raw silk bag are 4d6 long, sharp thorns. If a thorn is planted in the ground, one round later a 10 feet diameter by 10 feet high wickedly barbed thicket will sprout. If two thorns are planted within 10 feet of each other, a thick hedge wall will grow. Brushing against the hedge causes 1 point of damage, and any who try to pass through the hedge suffer 1d6 points of damage per foot, moving at only one foot per round, destroying clothing and damaging leather in the process. Size small creatures may subtract 1 point of damage per die, and size tiny creatures suffer only half damage. Chopping through the brush takes one round per foot; setting fire requires one round to light, then burns away a 5 feet section per round. Dispel magic can destroy the wall.  The magic disipates at dawn.


Aforgomon's Door Knocker
A brass door knocker featuring inscriptions of tentacular abominations and unspeakable runes.  It may be placed on a door and it will adhere to the door.  When knocked thrice the door will swing ajar.  Has a 1 in 6 chance of becoming drained per use.  Has a 1 in 20 chance of animating when it will attempt to bite the hand that knocks once and then try to run off.   If it is caught it will revert to being a door knocker.



Scoll of Spelling Mistakes

This scroll detects as magic but appears to be a treatise describing some feature of the current location. When read by a spell caster the reader immediately gets a headache and double vision for d6+3 turns. Thereafter the spell caster is subject to spelling mistakes in his spells at the rate of d6 per day. These take the form of single letter changes to the spell's name (and hence effects). So for Sleep read Sheep or Steep or Sleet. Initially the DM will make the substitutions and the player will only find out the effects when casting the spell (DMs must be generous!).

If a spell has no valid one-letter transformation it cannot be affected. After a week the player gains the ability to select the spelling mistakes and so to create some potentially powerful new spells (the DM still selects the effects though in line with the new spell title). This item gives lots of opportunity for fun and creativity among players and DMs (I never could persuade anyone to take "Piss Without Trace").

The DM can opt to allow a Remove Curse or more powerful equivalent to remove the effect at his discretion but I find that after a while other players are envious.


Book of the Past
A large ornate tome, decorated leather over wood cover plates, fine paper, says “my life” in common on cover with a handprint below– when person places his hand on the cover and holds it for 3 rounds, his detailed life history will be written in the book, updated every time he repeats. Will contain history of last person to own book. Will not work on the dead.


Coin of Larceny
The Coin of Larceny always appears as the most valuable coin in the realm that has been in circulation, i.e., it never looks newly minted. This bit of magical artifice must be used judiciously to prevent unwanted attention! The coin can be “spent” in any transaction that meets or exceeds the “value” of the coin once per day.  At midnight, the coin reappears in the owner’s purse. If the purse is stolen, the coin goes with it. A detect magic cast upon the coin will indicate that it is indeed magical. An identify spell will indicate that it is, in fact, a magical coin. The only way to discover the coin’s functioning is to use it properly, spending it in a situation where no change is expected. At midnight or the next day, the owner will discover the coin in his purse! 


Sword of the Phoenix
Flaming blade once per day +1 to hit.  +d4 Fire damage.

Cloak of Magic Shielding
+1 to saving throws versus magical attacks  When worn by a Magic-User or Illusionist it totally conceals all magical auras. All detections will fail.


Dawa’s Scrollcase of Safekeeping
An embossed brass scrollcase decorated with an abstract pattern.  The twin endcaps are inset with alternating ivory and malachite wedges. Any item placed in the scrollcase is impervious to normal fire, moisture, and the effects of aging. The scrollcase saves vs. magical damage from elemental effects and crushing at +4. 

Dog Whistle
This whistle carved from some canine bone allows the user to change the attitude of any canine within 30’ radius to non-hostile when blown (makes no sound). Hostile (attacking) canines will stop attacking and treat the character as unfriendly (snapping and barking at him), unfriendly dogs will be ambivalent to the presence of the character, usually moving away, and ambivalent canines will become friendly. Treat encountered wild animals as unfriendly, guard/attack dogs as hostile. Attacking any canine under the influence of the whistle automatically makes them hostile and dispels its effects. The effects last for 2d10 rounds. Can be used to call canines friendly to the user.


Dust of Darkness
A small pouch of Dust of Darkness contains enough for 2 uses. The user fills enough of his palm to not close it fully, then forcefully tosses it in the air. As it falls an area of 20 x 20' falls into total darkness for 15 minutes. A bag is worth 750gp.

Eye Stalk
A 1 inch diameter by 3 feet long segmented metal cylinder with glass lenses on both ends. The tube can be bent into nearly any twisted shape, including into a loose knot. No matter the shape, looking into the lens on one end, the user can see whatever the lens on the other end is facing, out to the limits of normal vision. It does not allow the use of infravision nor sight in the absence of light.

Fire Stones
xA leather bag with small pieces of lava rock. If the rock is placed with flammable material or kindling, then blown on or fanned, it will burst into flame. The Fire Stone will burn for 2 + 1d3 rounds and shed light and heat equal to a large candle. It can also be placed into an empty candleholder and used for temporary lighting. The stones are not otherwise flammable, and will not catch if fire is put to them. Fire Stones can be pulverized and used to create fiery effects if blown from the hand or from a tube, but will only burn for one round. A burning rock will cause 1 point of damage per round if placed against a living object, or 1d3+1 points of damage from a single powdered rock blown on a target.


Fleen's Reagent
Fleen's Reagent is a non-toxic gray powder useful in identifying magic items. Sprinkle a pinch on an item. If it uses negative magical energy, the powder turns black. It can be found for about 10 gold per pinch at better alchemist shops.

Gloves of Far Reaching
This normal looking pair of leather gloves confers the ability to manipulate objects at a distance, as per the spell telekinesis, at 12th level of ability.


Gloves of Wizardry
Made of the fine leather of a giant bat wing and sewn with the sinews of a constrictor snake – adds +3 bonus to a spell that requires a to hit roll once per encounter.  Reduces casting time of spells with somatic components by 1 (min. 1).  Must be used as a pair.



Hide of Transformation
What seems like an ordinary animal pelt with the face and paws attached is actually a magical item of transfiguring. When the skin is wrapped around the character, the wearer melds with the skin and transforms into the creature. All items carried or worn are absorbed as well. The character retains his mental attributes, level, and hit points, but acquires the AC, movement rate, and physical characteristics of the animal. Thus a badger could burrow, a giant crow could fly, and a tiger could climb trees. The character may use any natural attack forms the animal possesses such as teeth and claws, but with their own to-hit rolls. The transformation takes one round, and the character may return to his natural form at any time. A successful dispel magic spell will force the character back to his natural form.


 "'And that bursting blue flame -- I thought it looked familiar. It's a trick of the Stygian priests.'" -- Robert E. Howard: "Jewels of Gwahlur"

A magical dust that produces a blinding flash of blue fire. It is a common trick of Stygian priests and is also called flame dust. When thrown into the air it blazes in a flash of blue-white light.  When this blue dust is flung into the air, it mimics the effects of a glitterdust spell. It covers a 10 foot radius area no more than 10 feet away from the caster, and lasts for 3d4 rounds. Caster Level: 5th; Prerequisites: Craft Wondrous Item, glitterdust; Market Price: 2,100 gp; Weight: --.


It was but a dust I found in a Stygian tomb which I flung into your eyes. If I brush out their sight again, I will leave you to grope in darkness for the rest of your life!" -- Robert E. Howard: "The Scarlet Citadel"
This is a magical dust obtained from Stygian tombs. When flung into the eyes of an unsuspecting victim, it will cause temporary blindness. If a stronger dose is used, however, the blindness will be permanent.
Tsotha-Lanti, the Imperial Wizard of Koth, used Stygian Tomb-Dust against Amalrus, king of Ophir, in The Scarlet Citadel.
This sooty, black powder causes temporary magical blindness to all those in the area of effect (a 10 foot radius area no more than 10 feet away from the caster). Flinging the dust at opponents is a standard action.
If a creature's saving throw (Fortitude DC 16) is successful, he suffers no effects. If the roll fails, the creature is blinded and suffers the standard penalties for being blinded. Blindness persists each round until the victim succeeds at a saving throw, at which time the effect is negated.
An entire packet or blow tube must be used for each application. However, if a double dose is used, and the saving throw (Fortitude DC 18) is failed, the blindness is permanent.
Creating Stygian Tomb-Dust: Stygian Tomb-Dust must be taken from the crumbled corpse of a Stygian spellcaster of at least 12th level. One corpse is sufficient to produce 2d4 doses.
Caster Level: 7th; Prerequisites: Create Wondrous Item, ghoul touch; Market Price: 2,400 gp.



Mats of Travelling by Silbarak

There are rumoured to be seven of these items. Each one appears as a thin flexible soft leather circular (2' dia.) mat.
Each one has a specific symbol inlaid in gold in the center and six differing ones around the edge. If a character places on one the ground and stands upon it and touches one of the surrounding symbols by hand then he/she will be instantly transported (as teleport) to the mat which has the corresponding symbol in the centre, if it too is unfurled and on the ground.
Note that the mat must be on the ground - placing it on a table for instance will not work. Having two or more of these mats would allow the owner to teleport between the two and since they are portable one obvious use would be to leave one mat in a place of safety and carry the other(s) around . Consequently it is recommended that DM's exercise caution in allowing these items into their campaign.


Mimgot’s Anti-magic Girdle
Ming
This wide leather belt is made from the black hide of a beast of the Underdark, and the buckle is polished gold taken from a sacred mine. The powerful magic of this item confers a 15%% magic resistance to the wearer. Beneficial spells are also subject to disruption by this belt, as well as any cast by the wearer, including magic items that produce spell-like effects. This is a unique item.

Objects of Obsessiveness
These objects can take nearly any form, but are usually either painted ceramic plates, small statuettes, poorly painted cards, or cheaply bound folios of illuminated short stories. The first person to touch the cursed objects will suddenly ‘remember’ hearing about them, believing that they are parts of a larger collection, and that when gathered all together, the entire collection will be either extremely valuable or contain all the clues needed to unlock some ancient mystery of power. Nothing short of a remove curse or limited wish will convince the character otherwise, and they will obsessively seek to find the “missing” pieces of the collection. To this end they will haunt bazaars, trinket shops, and junk peddlers; seek oracles, and consult mystics and wise men; follow any rumor; and pour over dusty tomes in order to complete the whole. An enterprising DM can create a whole series of adventures surrounding these items, whether found by the characters or in the possession of another.


Parchment of Plagiarism
Parchment of Plagiarism appears to be an ordinary blank scroll. In fact it is very valuable to pilferers of magic.
By placing the parchment face down onto an open scroll a copy of that scroll is made. It can then be used by anyone who can ordinarily cast a spell from a scroll, with one exception. Since it is a copy there is a 5% chance it is flawed, and when cast it will not work correctly. Effects are subject to the DM's discretion.
Unlike a normal scroll that is rendered useless upon casting, the Parchment of Plagiarism can be used over and over again. It can only hold one spell at a time, but when it is used the spell disappears from the parchment and then it can be used to copy another scroll.






