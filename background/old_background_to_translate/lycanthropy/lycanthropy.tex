\documentclass[a4paper]{dnd5}

%\pagestyle{fancy}

\usepackage{wrapfig}

\newcommand\inc[1]{
 \includegraphics[width=0.22\textwidth, height=0.22\textwidth]{#1}
}

\begin{document}
\section*{Lycanthropy}
\begin{quotation}
Berthold of Regensburg
\end{quotation}


While it is widely held that Lycanthropy is a disease that has not been the result of our findings.  Instead we have come to conclude that it is caused by an contagion of 
the animus, a disease of the soul if you like.  More accurately, the wider scientific comunity believe it to be a form of parasitic demonic possession.  

Though individuals unfortunate enough to be infected are generally unaware of this.  The afliction generally presents itself as an overwhelming hunger for flesh (the projection of 
the parasitic other-worldly phage's desire for more animatic energy and in order to procreate).   These desires are particularly strong during lunar alignments when the connection between the host and the phage are at their strongest (the phage exists in some unknown other plane).


\begin{center}
 \includegraphics[width=0.45\textwidth]{lycaon.png}
\end{center}


Afflicted lycanthropes are normal humanoids who survive a lycanthrope's bite -- the lycanthropic phage must violate the victim's flesh in order to spread. Once the phage has possessed an individual, it duels with the humanoid for control of the mind and body.

All lycanthropes have certain elements in common beyond the expected basics (shapechanging, damage reduction/silver, etc.).

Lycanthropes attain unnatural vigour by feasting on the animus of others.  This vigour enhances their primal emotions and physical sensations, strengthens and repairs their bodies, and fuels their transformations, but it also burns out quickly, leaving the werewolf with a ravenous hunger for flesh that can never be completely sated. Even the most kind and gentle afflicted lycanthrope, one who has successfully fought for control over the beast, will always, always hunger for humah flesh - -an urge to indulge in their desires, to unleash the inner beast, and to feel hot blood on their lips. .

An afflicted lycanthrope is torn between its two natures. Even when the humanoid soul surrenders entirely, allowing the beast to consume it, the lycanthrope will always struggle with self-control. They usually lose control when they transform and seldom remember their actions once they change back. An afflicted lycanthrope can have two alignments -- one in humanoid form and one in bestial form. Regardless, the horrific truth is that ultimately in both the forms, the same mind drives the creature's actions. When the beast is in control, however, the human mind is consumed by carnal hunger and is stripped of its capacity for restraint. Heedless of ramifications, it simply takes everything it wants.

The perception of lycans is fairly uniform. Werewolves are universally feared, distrusted, and despised. There are two big reasons for this intense hatred – the fact that lycans are known to eat humanoids, and the disease known as lycanthropy.

There are many myths surrounding lycanthropy. It used to be thought that only humans could contract it, but it is known now that any humanoid can become infected. It is commonly thought that infected persons change shape on the night of a full moon, but in actuality the change is triggered when the infected person is placed into a high-stress situation – combat is a common trigger. Many people lose control when the change occurs, but those with a strong will have been known to stay lucid. No one, however, has been able to revert themselves back to their natural form by force of will – they must wait for a few minutes for the outbreak to end. 




\subsection{Prognosis}


The stages of Lycanthrope infection are:
\begin{itemize}
\item Stage 0: The curse is dormant.
\item Stage 1: 
\item Stage 2: The target has no allies.
\item Stage 3: While affected by stage 3, whenever the target becomes bloodied he or she may only make melee attacks, gains 5+level temporary hit points, does +2 damage in melee and gains regeneration 2+half level. If the target takes damage from silver weapons this regeneration does not take effect until next turn.
\item Stage 4: The target transforms into a werewolf at night when bloodied or when the moon is in the sky.
Check: at the end of each extended rest, the target makes a Nature check at any stage but 4.
Lower than easy DC: The curse increases by 1 stage.
Moderate DC: The stage of the curse decreases by 1.
\end{itemize}

\subsection{Werewolf Transformation}


Type: The targets type changes to magical beast (shapechanger).
Initiative Character’s Initiative +4 Senses Add Darksight 20
AC As Character; Fortitude As Character +4; Reflex As Character +2; Will As Character -6
HP As Character +10 per tier
Speed As Character +4
Traits
         Bestial Allies
Only Werewolves and Wolves are considered allies. All other individuals are considered foes.
·         Frenzy
Make a Moderate Wisdom check at -4. If the check is failed the character must act as follows:
1)      If there is a downed foe within move distance move towards that foe and Feast if possible.
2)      If there is an active foe move towards the nearest and attack if possible.

         Regeneration
The Character gains Regeneration of +5 +1/2 Character Level. This regeneration does not work on the round the character was damaged by a silver weapon or in the following round.

         Silver Vulnerability
Regeneration does not operate on the round the character is damaged by a silver weapon, or on the following round.
Standard Action

         Bite (standard; at will) Disease
Attack: Melee; + Character Level + ¼ Character Level vs. AC
Damage: 2d8 + 1d8/6 Character Levels + Strength Modifier damage.
Effect: The target must make a Moderate Endurance Check at the Character’s Level or contract the Lycanthopy Disease.

         Feast (standard; at will) Disease 
Prerequisite: The target must be at 0 hit points or lower and be unconscious or helpless.
Effect: Regain hit points as if you spent a healing surge.

Alignment evil Languages - 
Skills Str +3, Athletics +3, Endurance +3, Nature +3, Perception +3.








\subsection{Treatment}

\begin{flushright}
 \includegraphics[width=0.4\textwidth]{werewolf.png}
\end{flushright}

Lycanthropy is almost incurable.  It can however be managed.


If I remember correctly, lycanthropy is supposed to be far more insidious in Ravenloft than it is in other settings. By Ravenloft tradition, the "cure" for lycanthropy was to hunt down and kill the offending lycanthrope and conduct a ritual that would allow the afflicted lycanthrope a single save to rid him/herself of the disease. If they failed that save...

I think there were some spells printed later that were supposed to help suppress lycanthropy and help an afflicted character shrug it off by providing a bonus to the save, but I haven't scoured the expanded spell lists for a long time.

Ah yes the spell is called Suppress Lycanthropy. The Magical Item is quite rare and called something like the Silver Amulet of the Wolf maybe?

It's not impossible to cure lycanthropy in Ravenloft, just really difficult.  First step is to kill the natural lycanthrope that infected you. If the one who infected you was afflicted, you have to trace the affliction back to the natural one who started it. Only when the natural lycanthrope progenitor is dead can you try the cure. You need atonement, remove curse, and remove disease cast on you while in beast form, and then have one chance at a save. If the save fails, you can never be cured. (in that case, you'll be wanting the aforementioned Suppress Lycanthropy spells and Amulet of the Beast (silver version)).  

Lycanthropy is permanent until cured. Spells which remove diseases or curses are effective at curing lycanthropy. However, infected people are sometimes mistaken for true lycanthropes, and are sometimes killed before they can be treated. Though it is rumored that if someone is infected for long enough they will eventually become a true lycanthrope, there is no recorded instance of this happening.


Wolfssegen


\end{document}







