\documentclass[a4paper]{dnd5}
\usepackage{wrapfig}

\newcommand\inc[1]{
 \includegraphics[width=0.22\textwidth, height=0.22\textwidth]{#1}
}


\newcommand\origin{\textbf{Origin:}}
\newcommand\sigil{\textbf{Sigil:}}
\newcommand\words{\textbf{Words:}}

\pagestyle{fancy}

\newtoggle{DM}
%\toggletrue{DM}
\togglefalse{DM}



\iftoggle{DM}{\lhead{DM's Version}}{}


\begin{document}


\section*{Languages}

All Thules speak Thule. Some Thules in their outlying districts speak Brythinian, or Samartian, or even Orcish depending on where they live within that country and their occupations.

All elves speak elvish.
High elves, from Sylvannia, speak Aquilonian as well.
Wood elves from the forests south of Cheruskia speak Cheruskian and/or Brythynian.
Dark elves speak Arkadian (that is the language of their religious texts) and possibly Brythinian, Hibernian, Cheruskian as well.

All Dwarves speak Dwarven and Brythynian.

Brythinians typically only speak Brythynian. Travellers and those living near the borders might speak Thule, Cheruskian, Hibernian or Averoigne.

Stout halfings come from villages around Midheim and speak Brythinian, and either Cheruskian or Dwarven.
Lightfoots come from villages scattered through Hibernia and speak Hibernian and Brythinian.

Cheruskians speak Cheruskian.

All Hibernians speak Hibernian and most speak Brythinian as well.  

Aquillonian is the language of learning and trade around the middle seas.  In the north Brythynian has surplanted it for trade purposes.  Aquillonian is however, still, a very important language.

\end{document}
